% This is sigproc-sp.tex -FILE FOR V2.6SP OF ACM_PROC_ARTICLE-SP.CLS
% OCTOBER 2002
%
% It is an example file showing how to use the 'acm_proc_article-sp.cls' V2.6SP
% LaTeX2e document class file for Conference Proceedings submissions.
% 
%----------------------------------------------------------------------------------------------------------------
% This .tex file (and associated .cls V2.6SP) *DOES NOT* produce:
%       1) The Permission Statement
%       2) The Conference (location) Info information
%       3) The Copyright Line with ACM data
%       4) Page numbering
%
%  However, both the CopyrightYear (default to 2002) and the ACM Copyright Data
% (default to X-XXXXX-XX-X/XX/XX) can still be over-ridden by whatever the author
% inserts into the source .tex file.
% e.g.
% \CopyrightYear{2003} will cause 2003 to appear in the copyright line.
% \crdata{0-12345-67-8/90/12} will cause 0-12345-67-8/90/12 to appear in the copyright line.
%
%
%---------------------------------------------------------------------------------------------------------------
% It is an example which *does* use the .bib file (from which the .bbl file
% is produced).
% REMEMBER HOWEVER: After having produced the .bbl file,
% and prior to final submission,
% you need to 'insert'  your .bbl file into your source .tex file so as to provide
% ONE 'self-contained' source file.
%
% Questions regarding SIGS should be sent to
% Adrienne Griscti ---> griscti@acm.org
%
% Questions/suggestions regarding the guidelines, .tex and .cls files, etc. to
% Gerald Murray ---> murray@acm.org 
%
% For tracking purposes - this is V2.6SP - OCTOBER 2002


\documentclass[12pt]{article}

\setlength{\oddsidemargin}{0in}
\setlength{\evensidemargin}{0in}
\setlength{\topmargin}{0in}
\setlength{\headheight}{0in}
\setlength{\headsep}{0in}
\setlength{\textwidth}{6in}
\setlength{\textheight}{9in}
\setlength{\parindent}{0in} 

\usepackage{graphicx} %For jpg figure inclusion
\usepackage{times} %For typeface
\usepackage{epsfig}
\usepackage{color} %For Comments
%\usepackage[all]{xy}
\usepackage{float}
%\usepackage{subfigure} 
\usepackage{hyperref}
\usepackage{url}
\usepackage{parskip}

%% Elena's favorite green (thanks, Fernando!)
\definecolor{ForestGreen}{RGB}{34,139,34}
\definecolor{BlueViolet}{RGB}{138,43,226}
\definecolor{Coquelicot}{RGB}{255, 56, 0}
\definecolor{Teal}{RGB}{2,132,130}
%Uncomment this if you want to show work-in-progress comments
\newcommand{\comment}[1]{{\bf \tt  {#1}}}
% Uncomment this if you don't want to show comments
%\newcommand{\comment}[1]{}
\newcommand{\emcomment}[1]{\textcolor{ForestGreen}{\comment{Elena: {#1}}}}
\newcommand{\todo}[1]{\textcolor{blue}{\comment{To Do: {#1}}}}
\newcommand{\hfcomment}[1]{\textcolor{Teal}{\comment{Henry: {#1}}}}
\newcommand{\thcomment}[1]{\textcolor{Coquelicot}{\comment{Thomas: {#1}}}}
%%%%%%%%%%%%%%%%%%%%%%%%%%%%%%%%%%%%%%%%%%

\begin{document}
\pagestyle{plain}
%
% --- Author Metadata here ---
%\conferenceinfo{WOODSTOCK}{'97 El Paso, Texas USA}
%\setpagenumber{50}
%\CopyrightYear{2002} % Allows default copyright year (2002) to be
%over-ridden - IF NEED BE. 
%\crdata{0-12345-67-8/90/01}  % Allows default copyright data
%(X-XXXXX-XX-X/XX/XX) to be over-ridden. 
% --- End of Author Metadata ---

\title{Designing a Comparative Usability Study of Error Messages}
%\subtitle{[Extended Abstract \comment{DO WE NEED THIS?}]
%\titlenote{}}
%
% You need the command \numberofauthors to handle the "boxing"
% and alignment of the authors under the title, and to add
% a section for authors number 4 through n.
%
% Up to the first three authors are aligned under the title;
% use the \alignauthor commands below to handle those names
% and affiliations. Add names, affiliations, addresses for
% additional authors as the argument to \additionalauthors;
% these will be set for you without further effort on your
% part as the last section in the body of your article BEFORE
% References or any Appendices.

\author{
Henry Fellows, Thomas Hagen, Sean Stockholm, and Elena Machkasova \\
Computer Science Discipline \\
University of Minnesota Morris\\
Morris, MN 56267\\
fello056@morris.umn.edu,..., elenam@morris.umn.edu
}
\date{}
\maketitle
\thispagestyle{empty}

\section*{\centering Abstract}
Error messages are the only form of response that programmers get from malfunctioning programs. More experienced programmers often develop intuition about what error messages actually mean, but novices only have the content of the error message. Our research focuses on two functional programming languages in the Lisp family, their current or potential use in introductory CS classes, and specifically on the quality of their error messages for beginner CS students. 

The languages we will be comparing are a subset of Racket called beginning student language, and Clojure. Beginning student language is a language designed for introductory students using the “How to Design Programs 2” curriculum, with error messages that are designed for novices. Clojure is a Lisp built on top of the Java programming language which better supports concurrent and parallel programming and has been rapidly gaining popularity in industry. However, Clojure was not developed with beginner programmers in mind: its native error messages are often just Java error messages that don’t make sense to programmers without Java background. Our previous work with Clojure has built an alternative error messages system that we think may be more useful to beginner CS students. 

Our work is a part of the ClojurEd project which aims to use Clojure to teach an introductory CS course. One of the project goals is to provide introductory students with understandable error messages. We attempt to achieve this goal by rephrasing Java errors in terms that are familiar to new Clojure programmers. As a simple example, we replace references to specific Java numeric types, such as “int” and “double”, with just the term “number”. The next stage of the project is to evaluate how well the new messages work for beginner programmers. 

In order to evaluate how helpful our error messages are to beginners, we have designed a usability study comparing the proposed and standard Clojure error messages to each other and to Racket error messages. There are very few published usability studies of error messages and very few guidelines for evaluating quality of error messages, thus we have developed methodology for comparing error messages in two similar languages. We evaluate the effectiveness of our error messages by giving students, who are familiar with Racket, erroneous code in both Racket and Clojure. We will then observe how the information presented within the error messages helps them to correct the erroneous code. Effectiveness of error messages will primarily be measured by the number of code samples successfully corrected by students, as well as the number of attempts is takes a student to correct the errors. 

In this paper we present the details of the study and the approaches to developing code samples that allow us to compare error messages systems. We also present and discuss preliminary results of the usability study. 
\emcomment{This would need to be modified after we've finished the paper}
\newpage
\setcounter{page}{1}

\section{Introduction}\label{sec:intro}
	\hfcomment{Elena}
	\subsection{Goals}\label{sec:goals}
	\hfcomment{This needs to describe error messages briefly. Reference previous project.}
\section{Languages}\label{sec:lang}
	\hfcomment{Henry}
	\subsection{Overview of Lisp}\label{sec:lisp}
	Lisp is a family of programming languages that dates from 1958, created from Alonzo Church's lambda calculus by John McCarthy. Lisp was the first language where the internal representation of the program was in the same data structures the program was written in. \hfcomment{???} Lisp has a distinctive fully parenthesized prefix notation, where all code and data are written as expressions in the form:
	\begin{verbatim}
	(<function-name> <argument 1> <argument 2> ... <argument N>)
	\end{verbatim}
Almost all operations are functions. Addition, for instance, is a function that can be applied to any number of arguments.
	\begin{verbatim}
	(+ 2 4 6 8)
	-> 20
	\end{verbatim}
Note that \texttt{->} indicates the result of computations in the Clojure interpreter. Lists in lisp are commonly created by using the list function
 	\begin{verbatim}
	(list 2 4 6 8)
	-> '(2 4 6 8)
	\end{verbatim}
The return value \texttt{'(2 4 6 8)} uses the single quote \texttt{'} to indicate that the expression should not be evaluated (doing so would attempt to apply the number \texttt{2} as a function), but instead it should be treated as list \hfcomment{wrong, but close enough to the truth?}	
	
	Felleisen et al~\cite{Felleisen:2004} has made an excellent case for using a Lisp as a programming language in an introductory CS classes. Lisp offers a simple syntax and introduces students to modularity, abstraction, and data-driven program design while developing good programming practices by encouraging these program design principles. Teaching these concepts in introductory courses creates a foundation that later classes can build upon in  while teaching popular imperative and object-oriented languages.~\cite{Bieniusa:2008}.\hfcomment{grab from mics 2015 bib} 
	\subsection{Differences between Racket and Clojure}\label{sec:diff}

Racket (beginning student language) is designed/hobbled for beginners.

Clojure is built on the JVM.

Clojure is fast.

Clojure focuses on concurrency.
\section{Usability Study}\label{sec:study}
	\subsection{Objectives}\label{sec:obj}
	\hfcomment{Elena}
	\subsection{Experimental Setup}\label{sec:setup}
	\hfcomment{Thomas, Sean}
		\thcomment{Whats going to happen in terms of presenting to students, how they'll get the code, how we'll be using racket vs clojure. Clearly describing what we're doing technical to the point of what students are presented with, what we'll give them overview. Each student will have alternating examples presented. Website presentation to the students. What is being measured from this. Exit questions given to students.}

Outline:
Students will be given a Racket review
Students will be given a Clojure lesson
Students will be assigned an instructor
Each student will be presented with questions
The questions will be Racket and Clojure questions
The error messages of the Clojure questions will either be our error messages or Clojure's default error messages
The error messages of the Racket questions will always be default Racket
Students will be presented these questions through a website we have set up
The questions will be randomized in order and which student receives what type of errors
The errors students create and their current code will be captured via screencap when they evaluate their code and it returns an error
This will give us insight into how students act when presented with the same problem across different languages based on the set of errors that language uses.


	The participants in the study will be gathered into a common space at the start so that we can familiarize the students with Clojure as well as review Racket, which they should already be familar with at some level. The Racket section will cover basic functionality in Racket such as infix notation, higher order functions, function definition, and recursion. \hfcomment{An introduction to ...}Clojure introduction will follow, covering similar topics and how they translate from Racket outlining the underlying similarities with functional languages. Clojures lesson will cover the same curriculum in the syntax of Clojure, with the addition of introducing participants to Clojure's hashes as well. The overall goal of this instructional session is to not only educate the participants such that the problems to be presented to them will be solvable, but also to ensure a relatively similar level of familiarity with both Clojure and Racket. 
	After the educational part of the study, the participants will be assigned to a computer. They will be directed to a website we have constructed that will present them with problems to solve in both Racket and Clojure. The set of potential problems that any one participant could receive will be randomized and the type of error messages received will be randomized as well. Participants will also be allowed to use Racket's IDE and a Clojure project in Lighttable setup by us ahead of time. The participants will be allocated a timelimit of 30 minutes to answer the questions given. They are allowed to try as many times as needed on each questions and they are allowed to skip past questions with the possibility of coming back to them later on. Once the time limit has expired the participants will no longer be able to submit answers and they will be allowed to leave. After the question period of the study, we will now have a collection of screen captures from each participant detailing the process solving the problems. From these we can begin to gauge the significance of the error messages in their problem solving processes.
	
	

\section{Principles of Question Selection}\label{sec:select}
	\hfcomment{Henry + Elena}
	\subsection{Selecting Meaningful Accessible Questions}\label{sec:meaning}
	\hfcomment{Elena}
	\subsection{Question parallelism}\label{sec:parallel}
	\hfcomment{Henry}
	both features - don't use vectors, cond, no loop recur in clojure
\section{Examples}\label{sec:examples}
	\hfcomment{Sean, Thomas}

\section{Conclusion}\label{sec:conclusion}
	\hfcomment{Elena}

\bibliographystyle{acm}
\bibliography{mics2016}

% That's all folks!
\end{document}
