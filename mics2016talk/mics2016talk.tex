\documentclass{beamer}
\usetheme{Montpellier}
\usecolortheme{dolphin}

%\usepackage{graphicx} %For jpg figure inclusion
%\usepackage{times} %For typeface
%\usepackage{epsfig}
\usepackage{color} %For Comments
\usepackage{beamerthemeshadow}
%\usepackage[all]{xy}
%\usepackage{float}
%\usepackage{subfigure} 
%\usepackage{hyperref}
%\usepackage{url}
%\usepackage{parskip}
%\usepackage{multirow}

%% Elena's favorite green (thanks, Fernando!)
\definecolor{ForestGreen}{RGB}{34,139,34}
\definecolor{BlueViolet}{RGB}{138,43,226}
\definecolor{Coquelicot}{RGB}{255, 56, 0}
\definecolor{Teal}{RGB}{2,132,130}
% Uncomment this if you want to show work-in-progress comments
\newcommand{\comment}[1]{{\bf \tt  {#1}}}
% Uncomment this if you don't want to show comments
%\newcommand{\comment}[1]{}
\newcommand{\emcomment}[1]{\textcolor{ForestGreen}{\comment{Elena: {#1}}}}
\newcommand{\todo}[1]{\textcolor{blue}{\comment{To Do: {#1}}}}
\newcommand{\hfcomment}[1]{\textcolor{Teal}{\comment{Henry: {#1}}}}
\newcommand{\thcomment}[1]{\textcolor{Coquelicot}{\comment{Thomas: {#1}}}}
%%%%%%%%%%%%%%%%%%%%%%%%%%%%%%%%%%%%%%%%%%

\begin{document}
\title{Designing a Comparative Usability Study of Error Messages}
\date{April 23, 2016}

\begin{frame}
\frametitle{Designing a Comparative Usability Study of Error Messages}
{\centering
\noindent
Henry Fellows, Thomas Hagen, Sean Stockholm, \par
and Elena Machkasova \par

{\it 
Midwest Instruction and Computing Symposium\par
April 23, 2016\par}
}
\end{frame}
%frame

\begin{frame}
\frametitle{Table of contents}
\tableofcontents  
\end{frame}

\section{Introduction to the Project}

\begin{frame}
	\frametitle{Clojure in an introductory course}
	\begin{itemize}
		\item Developed in 2007 by Rich Hickey
		\item Member of the Lisp family
		\item Felleisen et al found Lisp languages to be useful in introductory courses
		\item Current UMM course uses a Lisp language
	\end{itemize}
\end{frame}

\begin{frame}
\frametitle{Motivations for the project}
	\begin{itemize}
		\item ClojurEd
			\begin{itemize}
				\item ongoing project at UMM
				\item introduce Clojure in an introductory course
			\end{itemize}
		\item Our work focuses on error messages in Clojure
			\begin{itemize}
				\item error messages are a useful learning tool
				\item focus on usability
			\end{itemize}
	\end{itemize}
\end{frame}

\section{Overview of Clojure}

\begin{frame}
\frametitle{Overview of Clojure}
	\begin{itemize}
  	 \item Dynamically typed
  	 \item Data types immutable by default
  	 \item Functional
  	 \item Runs on the Java Virtual Machine (JVM)
  	 \item Read-eval-print-loop (REPL)
  	 	\begin{itemize}
  	 	\item interactive environment
  	 	\item useful for development and debugging
  	 	\end{itemize}
	 \end{itemize}
\end{frame}

\begin{frame}[fragile]
\frametitle{Prefix notation}
	\begin{itemize}
  	  \item Clojure uses prefix notation
  	  \begin{itemize}
  	 	 \item parentheses
  	 	 \item parameters
  	 	 
  	 	 \texttt{(<function-name> <argument 1> <argument 2>)}
  	 	 \item \texttt{+} is a built-in function, not an operator
  	 	 \begin{verbatim}		
		 (+ 5 5)
		 -> 10
	     \end{verbatim}
	  \end{itemize}
   \end{itemize}
\end{frame}




\begin{frame}

\frametitle{Acknowledgments}
	Our research was sponsored by:
	\begin{itemize}
	\item HHMI
	\item UMN UROP
          \item Coginitect, Inc proivding funding for participants' compensation 
	\end{itemize}
	
	
	{\centering
	\noindent
	Thank you! \par
	Any questions? \par
	}
\end{frame}

\end{document}

