 \documentclass{beamer}
\usetheme{Montpellier}
\usecolortheme{dolphin}

%\usepackage{graphicx} %For jpg figure inclusion
%\usepackage{times} %For typeface
%\usepackage{epsfig}
\usepackage{color} %For Comments
\usepackage{beamerthemeshadow}
%\usepackage[all]{xy}
%\usepackage{float}
%\usepackage{subfigure} 
%\usepackage{hyperref}
%\usepackage{url}
%\usepackage{parskip}
%\usepackage{multirow}

%% Elena's favorite green (thanks, Fernando!)
\definecolor{ForestGreen}{RGB}{34,139,34}
\definecolor{BlueViolet}{RGB}{138,43,226}
\definecolor{Coquelicot}{RGB}{255, 56, 0}
\definecolor{Teal}{RGB}{2,132,130}
% Uncomment this if you want to show work-in-progress comments
\newcommand{\comment}[1]{{\bf \tt  {#1}}}
% Uncomment this if you don't want to show comments
%\newcommand{\comment}[1]{}
\newcommand{\emcomment}[1]{\textcolor{ForestGreen}{\comment{Elena: {#1}}}}
\newcommand{\todo}[1]{\textcolor{blue}{\comment{To Do: {#1}}}}
\newcommand{\hfcomment}[1]{\textcolor{Teal}{\comment{Henry: {#1}}}}
\newcommand{\thcomment}[1]{\textcolor{Coquelicot}{\comment{Thomas: {#1}}}}
%%%%%%%%%%%%%%%%%%%%%%%%%%%%%%%%%%%%%%%%%%

\begin{document}
\title{Designing a Comparative Usability Study of Error Messages}
\date{April 23, 2016}

\begin{frame}
\frametitle{Designing a Comparative Usability Study of Error Messages}
{\centering
\noindent
Henry Fellows, Thomas Hagen, Sean Stockholm, \par
and Elena Machkasova \par

{\it 
Midwest Instruction and Computing Symposium\par
April 23, 2016\par}
}
\end{frame}
%frame

\begin{frame}
\frametitle{Table of contents}
\tableofcontents  
\end{frame}

\section{Introduction to the project}

\begin{frame}
\frametitle{Our project}
	\begin{itemize}
		\item ClojurEd
			\begin{itemize}
				\item Ongoing project at UMM
				\item Goal: use Clojure in an introductory course
				\item Educators have found Lisp languages (e.g. Racket) to be useful in introductory courses
				\item Functional abstraction allows clearer abstraction: separating the data from the means of interacting with that data
				\item Prioritizes learning concepts, not a language
				\item Current UMM course uses a Lisp language 
			\end{itemize}
	\end{itemize}
\end{frame}


\begin{frame}
\frametitle{Current state of the project}
	\begin{itemize}
		\item Our work focuses on error messages in Clojure
			\begin{itemize}
				\item Error messages are a useful learning tool
				\item Primary means of communication
				\item Focus on usability
				\item Summer 2015: developed an alternative system of error messages
				\item Current goal: evaluate their effectiveness
				\item Would like to compare to usual Clojure and to Racket
			\end{itemize}
	\end{itemize}
\end{frame}

\section{Overview of Racket and Clojure}

\begin{frame}
\frametitle{Lisp}
	\begin{itemize}
		\item Clojure and Racket are Lisps
		\item Lisp is a functional dynamically typed language
		\item Data immutable by default
  			\item Read-eval-print-loop (REPL)
  	 		\begin{itemize}
  	 		\item interactive environment
  	 		\item useful for development and debugging
  	 	\end{itemize}
	\end{itemize}
\end{frame}

\begin{frame}[fragile]
\frametitle{Lisp prefix notation}
	\begin{itemize}
  	  \item Lisp uses prefix notation
  	  \begin{itemize}
  	 	 \item parentheses
  	 	 \item parameters
  	 	 
  	 	 \texttt{(<function-name> <argument 1> <argument 2>)}
  	 	 \item \texttt{+} is a built-in function, not an operator
  	 	 \begin{verbatim}		
		 (+ 5 5)
		 -> 10
	     \end{verbatim}
		\item Anonymous functions (lambdas) are a common idiom in lisp
		
	  \end{itemize}
   \end{itemize}
\end{frame}

\begin{frame}
\frametitle{Overview of Racket}
\emcomment{Possibly move to earlier}


\hfcomment{blerg. Not looking forward to this.}
\end{frame}

\begin{frame}
\frametitle{Overview of Clojure}
	\begin{itemize}
	\item Developed in 2007 by Rich Hickey
	\item Member of the Lisp family
  	\item Runs on the Java Virtual Machine (JVM)
	\item Used in industry, especially for parallelism and data science
		\begin{itemize}
			\item Horrific error messages
			\item Beginner tools are under development
	 \end{itemize}
	 \end{itemize}
\end{frame}




%\begin{frame}
%\frametitle{Differences between Racket and Clojure II}
%\end{frame}

\section{Design of the usability study} 


\begin{frame}
\frametitle{Study goals}
The goal of the usability study is to compare usability of error messages:
\begin{itemize}
\item Our error messages to the original Clojure.
\item Our error messages to Racket.
\end{itemize}

We were inspired by an approach developed by Marceau, Fisler, and Krishnamurthi for evaluation of Racket error messages: does an edit made after seeing an error move the participant towards or away from the correct solution?

\end{frame}

\begin{frame}
\frametitle{Experimental Setup}
	Usability study: students are asked to correct several code fragments in Clojure and in Racket. 
	\begin{itemize}
		\item Participant given a review of Racket
		\item Participant given a series of program fragments in Racket to correct
		\item Participant presented with a brief overview of Clojure
		\item Participant given a series of  program fragments in Clojure to correct (with either default Clojure messages or our messages) 
		\item The screen recorded at a regular intervals
		\item The participants asked a few questions at the end
	\end{itemize}
\end{frame}

\begin{frame}
\frametitle{Study Participants}
	\begin{itemize}
		\item Volunteer students
		\item Taken Racket introductory course
		\item No/little experience with Clojure
		\item Recruitment from department mailing list
		\item Students compensated for time thanks to gift from Cognitect, Inc.
	\end{itemize}
\end{frame}

\begin{frame}
\frametitle{Data Evaluation}
	\begin{itemize}
		\item Continuous screen capture allows extraction of numeric data
			\begin{itemize}
				\item Time to solve
				\item Iterations to solve
				\item Problems solved
			\end{itemize}
		\item Interview allows us to gauge perception of error messages
	\end{itemize}
\end{frame}


\begin{frame}
\frametitle{Question Selection}
	\begin{itemize}
		\item Testing the usability of similar programming languages
		\item Are you testing syntax, semantics, or error messages?
		\item Select elements of the languages that have similar syntax and semantics
		\item Sometimes you have to sacrifice idiomatic code for testable code
	\end{itemize}
\end{frame}

\begin{frame}
	\frametitle{Meaningful \& Accessible questions}
The code examples should: 
	\begin{itemize}
		\item be simple enough to understand the intent with 2-3 test examples. 
		\item have mistakes that a beginner would make (e.g. switched function arguments, a mistyped identifier...) 
		\item be simple to fix (challenging: may  be caused by multiple issues, and a beginner; beginners may make more complex changes than needed)
		\item use the same simple set of features in Racket and in Clojure (use \texttt{equal?} in Racket rather than \texttt{check-expect} since Clojure {\tt =} is roughly the same). 
%For instance, Racket uses a function \texttt{check-expect} for testing. Clojure also has a similar library (expectations), but it is more convoluted. Thus in order to equalize the experience in %the two languages we chose to use only = and \texttt{equals?} function for testing since they are the same in the two languages. 
	\end{itemize}
\end{frame}

\begin{frame}[fragile]
\frametitle{Example 1}
Racket version:
\begin{verbatim}
		(define (my-length elements)
  			(cond
 			 [(empty? elements) 0]
			 	 [else (+ 1 (my-length (first elements)))]))
\end{verbatim}
Clojure version:  
\begin{verbatim}
		(defn my-length [elements]
 			(if (empty? elements) 0 (+ 1 
                     (my-length (first elements)))))
\end{verbatim}
Test cases:
\begin{verbatim}
		(= (my-length '(5 4 3 2 1)) 5)
		(= (my-length '()) 0)
		(= (my-length '(1 3 5 7 9 11)) 6)
\end{verbatim}
\end{frame}


\begin{frame}[fragile]
\frametitle{Example 1: error messages}
Racket error:
\begin{verbatim}
		first: expects a non-empty list; given: 1	
\end{verbatim}

Original Clojure:
\begin{verbatim}
		IllegalArgumentException Don't know how to create ISeq  
from: java.lang.Long  clojure.lang.RT.seqFrom 
(RT.java:528)
\end{verbatim}

Our Clojure: 
\begin{verbatim}
		Error: In function first, the first argument 5 must be 
a sequence but is a number.
\end{verbatim}
\end{frame}

\begin{frame}[fragile]
\frametitle{Example 2}
Racket version:
\begin{verbatim}
		(define (select-even elements) 
		  (foldl (lambda (x y) (if (even? y) (cons x y) y)) 
                         '() elements))
\end{verbatim}
Clojure version:  
\begin{verbatim}
		(defn select-even [elements] 
		  (reduce (fn [x y] (if (even? x) (cons y x) y))  
                       '() elements))
\end{verbatim}
Test case:
\begin{verbatim}
		(= '(2 4 6 8) (select-even '(1 2 3 4 5 6 7 8 9)))
\end{verbatim}
\end{frame}


\begin{frame}[fragile]
\frametitle{Example 2: error messages}
Racket error:
\begin{verbatim}
		even?: expects integer, given '()
\end{verbatim}

Original Clojure:
\begin{verbatim}
java.lang.IllegalArgumentException: 
Argument must be an integer: 
clojure.lang.PersistentList$EmptyList@1
               core.clj:1355 clojure.core/even?	
\end{verbatim}

Our Clojure: 
\begin{verbatim}
		Error: In function even?, the first argument () must be
		an integer number but is a list.
\end{verbatim}
\end{frame}



\section{Conclusions and future work} 

\begin{frame}
\frametitle{Conclusion}
There isn't much literature about systematic evaluation of error messages.

Designing a comparative usability study is challenging.

We look forward to the results of the study.  

	%Never do a usability study on programming languages.
\end{frame}

\begin{frame}
\frametitle{Acknowledgments}
	Our research was sponsored by:
	\begin{itemize}
	\item HHMI
	\item UMN UROP
          \item Coginitect, Inc providing funding for participants' compensation 
	\end{itemize}
	
	
	{\centering
	\noindent
	Thank you! \par
	Any questions? \par
	}
\end{frame}

\end{document}

