\documentclass{beamer}
\usetheme{Montpellier}
\usecolortheme{dolphin}

%\usepackage{graphicx} %For jpg figure inclusion
%\usepackage{times} %For typeface
%\usepackage{epsfig}
\usepackage{color} %For Comments
\usepackage{beamerthemeshadow}
%\usepackage[all]{xy}
%\usepackage{float}
%\usepackage{subfigure} 
%\usepackage{hyperref}
%\usepackage{url}
%\usepackage{parskip}
%\usepackage{multirow}

%% Elena's favorite green (thanks, Fernando!)
\definecolor{ForestGreen}{RGB}{34,139,34}
\definecolor{BlueViolet}{RGB}{138,43,226}
\definecolor{Coquelicot}{RGB}{255, 56, 0}
\definecolor{Teal}{RGB}{2,132,130}
% Uncomment this if you want to show work-in-progress comments
\newcommand{\comment}[1]{{\bf \tt  {#1}}}
% Uncomment this if you don't want to show comments
%\newcommand{\comment}[1]{}
\newcommand{\emcomment}[1]{\textcolor{ForestGreen}{\comment{Elena: {#1}}}}
\newcommand{\todo}[1]{\textcolor{blue}{\comment{To Do: {#1}}}}
\newcommand{\hfcomment}[1]{\textcolor{Teal}{\comment{Henry: {#1}}}}
\newcommand{\thcomment}[1]{\textcolor{Coquelicot}{\comment{Thomas: {#1}}}}
%%%%%%%%%%%%%%%%%%%%%%%%%%%%%%%%%%%%%%%%%%

\begin{document}
\title{Designing a Comparative Usability Study of Error Messages}
\date{April 23, 2016}

\begin{frame}
\frametitle{Designing a Comparative Usability Study of Error Messages}
{\centering
\noindent
Henry Fellows, Thomas Hagen, Sean Stockholm, \par
and Elena Machkasova \par

{\it 
Midwest Instruction and Computing Symposium\par
April 23, 2016\par}
}
\end{frame}
%frame

\begin{frame}
\frametitle{Table of contents}
\tableofcontents  
\end{frame}

\section{Introduction to the project}

\begin{frame}
\frametitle{Motivations and current state of the project}
	\begin{itemize}
		\item ClojurEd
			\begin{itemize}
				\item ongoing project at UMM
				\item goal: use Clojure in an introductory course
				\item Felleisen et al found Lisp languages to be useful in introductory courses
				\item Current UMM course uses a Lisp language 
			\end{itemize}
		\item Our work focuses on error messages in Clojure
			\begin{itemize}
				\item error messages are a useful learning tool
				\item focus on usability
				\item summer 2016: developed an alternative system of error messages
				\item current goal: evaluate their effectiveness
				\item would like to compare to usual Clojure and to Racket
			\end{itemize}
	\end{itemize}
\end{frame}

\section{Overview of Racket and Clojure}

\begin{frame}
\frametitle{Lisp}
\begin{itemize}
	\item Clojure and Racket are Lisps
	\item Lisp is a functional dynamically typed language
	\item Data immutable by default
  	 \item Read-eval-print-loop (REPL)
  	 	\begin{itemize}
  	 	\item interactive environment
  	 	\item useful for development and debugging
  	 	\end{itemize}
\end{itemize}
\end{frame}

\begin{frame}[fragile]
\frametitle{Lisp prefix notation}
	\begin{itemize}
  	  \item Clojure uses prefix notation
  	  \begin{itemize}
  	 	 \item parentheses
  	 	 \item parameters
  	 	 
  	 	 \texttt{(<function-name> <argument 1> <argument 2>)}
  	 	 \item \texttt{+} is a built-in function, not an operator
  	 	 \begin{verbatim}		
		 (+ 5 5)
		 -> 10
	     \end{verbatim}
	  \end{itemize}
   \end{itemize}
\end{frame}

\begin{frame}
\frametitle{Overview of Clojure}
\emcomment{Some of this can be moved to differences between Racket and Clojure, some eliminated}
	\begin{itemize}
  	 %\item Dynamically typed
  	 %\item Data types immutable by default
  	 %\item Functional
		\item Developed in 2007 by Rich Hickey
		\item Member of the Lisp family
  	 \item Runs on the Java Virtual Machine (JVM)

	\item \emcomment{Add more Clojure-specific features}
	 \end{itemize}
\end{frame}


\begin{frame}
\frametitle{Overview of Racket}
\emcomment{Possibly move to earlier}
\end{frame}

%\begin{frame}
%\frametitle{Differences between Racket and Clojure II}
%\end{frame}

\section{Design of the usability study} 

\begin{frame}
\frametitle{Experimental Setup}
\begin{itemize}
%\item Participant will be assigned an instructor
\item Participant will be given a review of Racket
\item Participant will be given a series of program fragments in Racket to correct
\item Participant will be presented with a brief overview of Clojure
\item Participant will be given a series of  program fragments in Clojure to correct (with either default Clojue messages or our messages) 
\item The screen will be recorded at a regular intervals
\item The participants will be asked a few questions at the end
\end{itemize}
\end{frame}

\begin{frame}
\frametitle{Study Participation}
\end{frame}

\begin{frame}
\frametitle{Languages}
\end{frame}

\begin{frame}
\frametitle{Testing}
\end{frame}

\begin{frame}
\frametitle{Data Evaluation}
\end{frame}

\begin{frame}
	\frametitle{Meaningful \& Accessible questions}
The code examples should: 
	\begin{itemize}
		\item be simple enough to understand the intent with 2-3 test examples. 
		\item have mistakes that a beginner would make (e.g. switched function arguments, a mistyped identifier...) 
		\item be simple to fix (challenging: may  be caused by multiple issues, and a beginner; beginners may make more complex changes than needed)
		\item use the same simple set of features in Racket and in Clojure (use \texttt{equal?} in Racket rather than \texttt{check-expect} since Clojure {\tt =} is roughly the same). 
%For instance, Racket uses a function \texttt{check-expect} for testing. Clojure also has a similar library (expectations), but it is more convoluted. Thus in order to equalize the experience in %the two languages we chose to use only = and \texttt{equals?} function for testing since they are the same in the two languages. 
	\end{itemize}
\end{frame}

\begin{frame}
\frametitle{Question parallelism}
\end{frame}


\begin{frame}
\frametitle{Question Selection}
\end{frame}


\begin{frame}
\frametitle{Example I}
\end{frame}

\begin{frame}
\frametitle{Example ||}
\end{frame}

\section{Conclusions and future work} 

\begin{frame}
\frametitle{Conclusion}
There isn't much literature about systematic evaluation of error messages.

Designing a comparative usability study is challenging.

We look forwr 

	%Never do a usability study on programming languages.
\end{frame}

\begin{frame}
\frametitle{Acknowledgments}
	Our research was sponsored by:
	\begin{itemize}
	\item HHMI
	\item UMN UROP
          \item Coginitect, Inc proivding funding for participants' compensation 
	\end{itemize}
	
	
	{\centering
	\noindent
	Thank you! \par
	Any questions? \par
	}
\end{frame}

\end{document}

